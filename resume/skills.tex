\cvsection{Computational Skills}
\cvsubsection{Programming fluency \& Software familiarity}
	\begin{cvskills}
		\cvskill{Languages}{Python (Scikit-Learn, Boto3, Numpy, Scipy, Pandas, Geopandas, Matplotlib, SQLAlchemy), PostgreSQL, PostGIS, FORTRAN90, HTML/CSS, Julia, Shell}
		\cvskill{Software}{AWS (EC2, ECS, S3, Glacier, Cloudwatch, Dynamodb, RDS, Kinesis), Docker, Git, Jupyter-Framework, Continuous Integration (Jenkins, Travis), MongoDB}
		\cvskill{Skills}{Unix, Bash scripting and basic system administration. Experience running parallel codes on HPC clusters of various architectures.}
		\cvskill{Research}{Expert user of computational chemistry research software such as VASP, DLPOLY, LAMMPS, and DLMONTE.}
		\cvskill{Other}{Holder of valid UK security clearance (SC)}
	\end{cvskills}
\cvsubsection{Open source software development}
    \begin{cvskillsy}
		\cvskill{\hspace{0.2cm}pythonmaps}{pythonmaps is a project to produce eye catching geospatial data visualisations using python. I have 21,600 followers on twitter, where I share these visualisations and articles outlining how to generate these visualisations. I will be delivering a workshop at the scipy conference in Austin this year where I will outline how to correctly visualise geospatial data. I have also recently partnered with Locate Press LLC to produce a geospatial data visualisation book.}
        \cvskill{\vspace{0.06cm}surfinpy}{surfinpy is an open-source Python library to facilitate the analysis and visualisation of large scale simulation data. surfinpy was originally published in the Journal of Open Source Software (Symington et al., J. Open. Source Soft. 4, 1210, 2019) and a followup paper outlining recent developments was published recently (Tse et al. J. Open. Source Soft. 7, 4014, 2022). The code has been used in five pieces of peer reviewed research.}
        \cvskill{polypy}{polypy is an open-source Python library designed to analyse molecular dynamics simulation data. polypy is built to read large datasets associated with molecular dynamics trajectories and from these produce insightful statistical information. polypy has been published in the Journal of Open Source Software (Symington, J. Open. Source Soft. 6, 59, 2021) and has been used in five pieces of peer reviewed research.}
	\end{cvskillsy}
